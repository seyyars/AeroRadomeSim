\documentclass[11pt,a4paper]{article}

\usepackage[margin=1in]{geometry}
\usepackage{amsmath,amssymb}
\usepackage{graphicx}
\usepackage{booktabs}
\usepackage[hidelinks]{hyperref}

\title{AeroRadomeSim: Numerical simulation of Mach 0--3 aerothermodynamics\\
and temperature-dependent radome RF transmission}

\author{Seyed Arsham Asgari%
\thanks{Independent researcher. Email: s.a.asg2000@gmail.com}}

\date{\today}

\begin{document}
\maketitle

\begin{abstract}
Airborne radomes experience significant aerothermodynamic heating at
moderate Mach numbers. The resulting temperature rise modifies the
temperature-dependent electromagnetic properties of the radome material
and degrades RF transmission. This work presents \emph{AeroRadomeSim},
a lightweight open-source Python tool that couples a simple
compressible-flow aerothermodynamic model with a temperature-dependent
RF transmission model for canonical radome configurations over Mach
numbers from $0$ to $3$. The software is intended for rapid parametric
studies, early-stage design screening, and educational use.
AeroRadomeSim is distributed under an open-source license, with source
code available on GitHub and a citable snapshot archived on Zenodo
(doi:\href{https://doi.org/10.5281/zenodo.17817671}{10.5281/zenodo.17817671}).
\end{abstract}

\section{Introduction}
Airborne antenna systems are commonly shielded from the external flow
by dielectric radomes. While the radome protects the antenna
mechanically, it also introduces RF transmission loss and boresight
error, both of which depend on the material permittivity, thickness,
and internal temperature. At flight conditions corresponding to Mach
numbers between $0$ and $3$, convective and compressible heating can
raise radome wall temperatures well above ambient conditions,
modifying the dielectric response of the material and, consequently, the
RF performance.

High-fidelity analysis of coupled aerothermodynamics and RF propagation
typically relies on large commercial CFD and full-wave electromagnetic
solvers. Such tools are powerful but also computationally expensive and
often inaccessible to students or early-stage design studies.
There is therefore value in small, script-level tools that implement
simplified but physically meaningful models and can be used for rapid
parametric exploration.

In this context, the main objectives of AeroRadomeSim are:
\begin{itemize}
  \item to provide a transparent implementation of a coupled
        aerothermodynamic and RF transmission model for canonical
        radome configurations over Mach 0--3;
  \item to enable fast parametric sweeps over Mach number, altitude,
        material properties, and wall thickness;
  \item to serve as a reproducible reference implementation that can be
        extended or embedded in larger analysis workflows.
\end{itemize}

\section{Physical and numerical model}
\label{sec:model}

\subsection{Aerothermodynamic model}
The aerothermodynamic model in AeroRadomeSim is based on a
compressible-flow formulation with simple correlations for stagnation
or recovery temperature at the radome wall. Given a free-stream Mach
number $M_{\infty}$, static temperature $T_{\infty}$, and ratio of
specific heats $\gamma$, the adiabatic stagnation temperature is
\begin{equation}
  T_0 = T_{\infty} \left( 1 + \frac{\gamma - 1}{2} M_{\infty}^2 \right).
\end{equation}
A recoverable wall temperature can be expressed as
\begin{equation}
  T_w = T_{\infty} + r \left( T_0 - T_{\infty} \right),
\end{equation}
where $r$ is a recovery factor (typically between $0.8$ and $1.0$)
that may depend on the local boundary-layer regime.

For the purposes of this tool, we assume a spatially uniform
equivalent wall temperature $T_w$ representative of the most
thermally loaded region of the radome. This approximation is
appropriate for quick estimates of RF performance degradation with
increasing Mach number and is clearly documented in the code for
users who wish to replace it with a more detailed thermal model.

\subsection{Temperature-dependent RF transmission}
Given a radome wall of thickness $d$ and complex relative permittivity
$\varepsilon_r(T) = \varepsilon'(T) - \mathrm{j}\,\varepsilon''(T)$, the
normal-incidence power transmission coefficient $|T|^2$ of a
single-layer dielectric slab can be computed using standard
transmission-line expressions. AeroRadomeSim adopts a parametric model
for the temperature dependence of the real relative permittivity and
loss tangent:
\begin{align}
  \varepsilon'(T) &= \varepsilon'_0 + a_{\varepsilon} (T - T_{\mathrm{ref}}), \\
  \tan\delta(T)   &= \tan\delta_0 + a_{\delta} (T - T_{\mathrm{ref}}),
\end{align}
where $\varepsilon'_0$ and $\tan\delta_0$ are the properties at a
reference temperature $T_{\mathrm{ref}}$, and $a_{\varepsilon}$,
$a_{\delta}$ are user-specified linear coefficients.
The complex propagation constant in the material is then
\begin{equation}
  k(T) = \frac{2\pi f}{c_0} \sqrt{\varepsilon_r(T)},
\end{equation}
with $f$ the RF frequency and $c_0$ the speed of light in vacuum.
From this, the tool computes the reflection and transmission
coefficients at each interface and the overall transmission magnitude.

\subsection{Numerical implementation}
All equations described above are implemented in vectorized form using
standard scientific Python libraries. The user supplies one or more
Mach numbers, material parameters, and RF frequencies, and the code
evaluates:
\begin{enumerate}
  \item the wall temperature $T_w(M_{\infty})$ for each Mach number,
  \item the corresponding dielectric properties $\varepsilon'(T_w)$ and
        $\tan\delta(T_w)$,
  \item the transmission magnitude $|T|^2$ as a function of Mach number
        and/or frequency.
\end{enumerate}
The implementation emphasizes clarity over extreme performance, with
all intermediate quantities exposed and documented so that users can
verify or modify the model.

\section{Software description}
\label{sec:software}

\subsection{Architecture and dependencies}
AeroRadomeSim is implemented in Python and currently consists of a
small set of modules that separate the aerothermodynamic model, RF
transmission calculations, and plotting utilities. The core
dependencies are:
\begin{itemize}
  \item \texttt{numpy} for numerical arrays and vectorized operations;
  \item \texttt{scipy} for selected numerical utilities;
  \item \texttt{matplotlib} for producing publication-quality plots.
\end{itemize}
The source tree includes example scripts and basic tests that
demonstrate typical usage patterns.

\subsection{Typical workflow}
A typical analysis workflow with AeroRadomeSim consists of:
\begin{enumerate}
  \item specifying flight conditions (Mach number range, altitude, and
        ambient temperature) and radome material parameters;
  \item evaluating the wall-temperature model over the Mach range;
  \item computing RF transmission for one or more RF frequencies;
  \item plotting wall temperature versus Mach and transmission loss
        versus Mach or temperature.
\end{enumerate}
The repository provides ready-to-run example scripts that generate
such plots with a few lines of user code.

\section{Example use case}
\label{sec:example}
As an illustrative example, consider a radome material with
properties $\varepsilon'_0 = 4.0$ and $\tan\delta_0 = 0.002$ at
$T_{\mathrm{ref}} = 300~\text{K}$, with mild linear temperature
sensitivities. Figure~\ref{fig:temp_mach} shows the predicted wall
temperature as a function of Mach number for a given altitude, based
on the aerothermodynamic model in Section~\ref{sec:model}. The
corresponding RF transmission loss at a fixed frequency is shown in
Figure~\ref{fig:loss_mach}.

% TODO: replace file names with actual figures generated by the code.
\begin{figure}[t]
  \centering
  \includegraphics[width=0.7\textwidth]{figures/temperature_vs_mach.pdf}
  \caption{Example wall-temperature prediction as a function of Mach
  number for a representative flight condition.}
  \label{fig:temp_mach}
\end{figure}

\begin{figure}[t]
  \centering
  \includegraphics[width=0.7\textwidth]{figures/transmission_loss_vs_mach.pdf}
  \caption{Example RF transmission loss through the radome as a
  function of Mach number, illustrating the impact of thermally induced
  changes in dielectric properties.}
  \label{fig:loss_mach}
\end{figure}

These results demonstrate how even a simplified model can quantify the
sensitivity of RF performance to flight condition and material
temperature dependence. Because all parameters are user-configurable,
the same workflow can be applied to alternative materials and operating
frequencies.

\section{Impact and reuse potential}
AeroRadomeSim is intentionally small, with a focus on clarity and
reproducibility. The main anticipated use cases are:
\begin{itemize}
  \item \textbf{Early-stage design studies}, where engineers wish to
        quickly screen radome materials or wall thicknesses over a
        range of Mach numbers without running full CFD or EM solvers.
  \item \textbf{Sensitivity and uncertainty analyses}, in which the
        effect of uncertain material properties or temperature
        dependencies on RF performance can be explored through
        parametric sweeps.
  \item \textbf{Education and training}, as a transparent example for
        students learning about aerothermodynamic heating,
        temperature-dependent material properties, and basic RF
        transmission modeling.
\end{itemize}
The code is organized to facilitate extension: users can introduce
alternative correlations for wall temperature, more complex temperature
dependencies for permittivity, or multi-layer radome models while
reusing the existing workflow and plotting utilities.

\section*{Code availability}
The AeroRadomeSim source code is openly available on GitHub at:
\begin{center}
  \url{https://github.com/seyvars/AeroRadomeSim}
\end{center}
A frozen, citable snapshot of version~1.0.0 is archived on Zenodo under
the DOI:
\begin{center}
  \url{https://doi.org/10.5281/zenodo.17817671}
\end{center}

\section*{Acknowledgements}
The author thanks the open-source scientific Python community for
providing the core libraries on which AeroRadomeSim is built.

\begin{thebibliography}{9}

\bibitem{radome_book}
A.~Author,
\newblock \emph{Title of a representative radome or RF design reference},
\newblock Publisher, Year.

\bibitem{aero_book}
B.~Author,
\newblock \emph{Title of a representative compressible-flow/aerodynamics text},
\newblock Publisher, Year.

% Replace the placeholder items above with real references relevant
% to radome design, aerothermodynamics, and RF transmission modeling.

\end{thebibliography}

\end{document}

Add report main.tex
