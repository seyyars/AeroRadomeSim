\documentclass[11pt,a4paper]{article}

\usepackage[margin=1in]{geometry}
\usepackage{amsmath,amssymb}
\usepackage{graphicx}
\usepackage{booktabs}
\usepackage[hidelinks]{hyperref}

\title{AeroRadomeSim: Lightweight numerical simulation of Mach 0--3\\
aerothermodynamics and temperature-dependent radome RF transmission}

\author{Seyed Arsham Asgari%
\thanks{Independent researcher. Email: s.a.asg2000@gmail.com}}

\date{\today}

\begin{document}
\maketitle

\begin{abstract}
Airborne radomes protect antenna systems from the external flow but also
introduce RF transmission loss and boresight error. At moderate Mach
numbers, compressible-flow and boundary-layer heating can raise radome
wall temperatures significantly above ambient conditions, modifying the
temperature-dependent dielectric properties of the material and further
degrading RF performance. High-fidelity assessment of this coupled
aerothermodynamic--electromagnetic problem typically relies on large
commercial CFD and full-wave solvers that are computationally expensive
and often inaccessible during early design or educational use.

This article presents \emph{AeroRadomeSim}, a small open-source Python
tool that couples a simple compressible-flow aerothermodynamic model to
a temperature-dependent transmission-line model of normal-incidence RF
propagation through a single-layer dielectric radome. The software is
designed for rapid parametric studies over Mach number, altitude,
material parameters, and wall thickness in the Mach~0--3 regime.
We describe the governing equations, numerical implementation, and
software design, and illustrate typical use cases through parametric
sweeps of dynamic pressure, wall temperature, and transmission loss.
Although intentionally lightweight, AeroRadomeSim provides a transparent
reference implementation that can be extended toward more complex
radome configurations and used as a teaching tool for coupled
aerothermodynamic and RF analysis.
\end{abstract}

\section{Introduction}
Airborne and ground-based antenna systems are frequently enclosed by
dielectric radomes that provide mechanical and environmental protection.
The radome must satisfy conflicting requirements: it should be
structurally robust while introducing minimal RF transmission loss and
boresight error over the operational frequency band and look angles.
Both performance metrics depend sensitively on the radome thickness,
dielectric properties, and internal temperature.

At flight conditions corresponding to Mach numbers between $0$ and $3$,
compressible-flow and boundary-layer heating can raise external skin
temperatures substantially above ambient values. For typical radome
materials---fiber-reinforced composites or ceramics---the real part of
the relative permittivity and the dielectric loss tangent exhibit a
non-negligible dependence on temperature. Consequently, aerothermal
heating feeds back on RF performance via temperature-dependent material
properties.

High-fidelity analysis of this coupled problem requires either (i) a CFD
or aerothermal solver to predict the spatially varying temperature
distribution followed by a full-wave electromagnetic simulation
(FDTD/FEM/MoM) on the resulting geometry and material map, or (ii)
fully coupled multi-physics commercial environments. These workflows
are accurate but computationally demanding, require detailed geometry,
and are not ideal for early design-space exploration, educational use,
or quick sensitivity studies.

In contrast, there is value in small script-level tools that implement
simplified but physically meaningful models and can be used to:
\begin{itemize}
  \item assess first-order trends of wall temperature and transmission
        loss with Mach number, altitude, and material properties;
  \item support quick trade studies between candidate radome materials;
  \item serve as transparent reference implementations that can be
        extended or embedded within larger analysis workflows.
\end{itemize}

The main contributions of the present work are:
\begin{enumerate}
  \item a compact, open-source Python implementation that couples a
        compressible-flow aerothermodynamic model to a
        temperature-dependent dielectric transmission model for a
        single-layer radome over Mach~0--3;
  \item a description of the governing equations and numerical
        implementation sufficient to reproduce and extend the code;
  \item a set of parametric case studies illustrating how the tool can
        be used for early-stage design screening and educational
        demonstrations.
\end{enumerate}

\section{Governing equations and models}
\label{sec:models}

\subsection{Aerothermodynamic model}
We consider a canonical external flow characterized by a free-stream
Mach number $M_{\infty}$, static temperature $T_{\infty}$, static
pressure $p_{\infty}$, and ratio of specific heats $\gamma$. For the
Mach range $0 \le M_{\infty} \le 3$ and typical altitudes below
$\sim 20~\text{km}$, an ideal-gas, calorically perfect model is
adequate. The adiabatic stagnation temperature is then
\begin{equation}
  T_0 = T_{\infty} \left( 1 + \frac{\gamma - 1}{2} M_{\infty}^2 \right).
  \label{eq:stagnationT}
\end{equation}
The dynamic pressure is
\begin{equation}
  q = \tfrac12 \rho_{\infty} U_{\infty}^2
    = \tfrac12 \gamma p_{\infty} M_{\infty}^2,
\end{equation}
where $\rho_{\infty}$ and $U_{\infty}$ denote free-stream density and
speed, respectively.

To account for boundary-layer recovery, we introduce a recovery factor
$r \in [0.8,1.0]$ and define a recoverable wall temperature
\begin{equation}
  T_w = T_{\infty} + r\,(T_0 - T_{\infty}),
  \label{eq:wallT}
\end{equation}
with $r$ chosen by the user to reflect the local flow regime
(e.g.\ $r\approx 0.9$ for turbulent flow over a blunt nose).
In AeroRadomeSim we approximate the external wall of the radome by a
single spatially uniform equivalent temperature $T_w$ representative of
the most thermally loaded region. This approximation is explicitly
documented and is adequate for quick sensitivity studies; users wishing
to incorporate more detailed aerothermal models can replace
Eq.~\eqref{eq:wallT} by a more sophisticated temperature field while
reusing the RF transmission module described below.

\subsection{Temperature-dependent dielectric model}
The radome wall is modeled as a homogeneous dielectric slab of thickness
$d$ and complex relative permittivity
$\varepsilon_r(T) = \varepsilon'(T) - j\varepsilon''(T)$.
We adopt a linear temperature dependence for the real permittivity and
loss tangent:
\begin{align}
  \varepsilon'(T) &= \varepsilon'_0 + a_{\varepsilon} (T - T_{\mathrm{ref}}),
  \label{eq:epsT}
  \\
  \tan\delta(T)   &= \tan\delta_0 + a_{\delta} (T - T_{\mathrm{ref}}),
  \label{eq:tandT}
\end{align}
where $\varepsilon'_0$ and $\tan\delta_0$ denote the properties at a
reference temperature $T_{\mathrm{ref}}$, and
$a_{\varepsilon}, a_{\delta}$ are user-specified linear coefficients
representing first-order thermal sensitivities.
The imaginary part of the permittivity is reconstructed via
\begin{equation}
  \varepsilon''(T) = \varepsilon'(T)\,\tan\delta(T).
\end{equation}
Within the temperature range encountered for Mach~0--3 flight, this
simple linear model is sufficient to capture monotonic degradation
trends. More complex dependencies can be incorporated without changing
the interface of the RF transmission module.

\subsection{Normal-incidence transmission-line model}
For a plane wave incident normally on a single-layer dielectric slab of
thickness $d$ embedded in free space, the complex transmission
coefficient $T$ can be computed using standard transmission-line
expressions. Given angular frequency $\omega = 2\pi f$ and free-space
propagation constant $k_0 = \omega / c_0$, the propagation constant in
the slab at wall temperature $T_w$ is
\begin{equation}
  k(T_w) = k_0 \sqrt{\varepsilon_r(T_w)}.
\end{equation}
Let $\eta_0$ denote the free-space impedance and $\eta_d$ the impedance
of the dielectric layer at $T_w$. The input impedance seen at the
entrance of the slab is
\begin{equation}
  Z_{\mathrm{in}} = \eta_d
    \,\frac{\eta_0 + j\eta_d \tan(k(T_w)\,d)}
           {\eta_d + j\eta_0 \tan(k(T_w)\,d)}.
\end{equation}
The resulting complex reflection and transmission coefficients at
normal incidence are
\begin{align}
  \Gamma &= \frac{Z_{\mathrm{in}} - \eta_0}{Z_{\mathrm{in}} + \eta_0}, \\
  T      &= 1 + \Gamma,
\end{align}
and the power transmission coefficient is $|T|^2$.
In AeroRadomeSim we evaluate these expressions over user-defined
frequency ranges and report either $|T|^2$ or transmission loss in
decibels,
\begin{equation}
  L_T = -10\log_{10} |T|^2.
\end{equation}

\section{Numerical implementation and software design}
AeroRadomeSim is implemented in Python and currently consists of a small
set of modules that separate the aerothermodynamic model, dielectric
model, and RF transmission calculations. The main dependencies are:
\begin{itemize}
  \item \texttt{numpy} for numerical arrays and vectorised operations;
  \item \texttt{scipy} for selected mathematical utilities;
  \item \texttt{matplotlib} for generating publication-quality plots.
\end{itemize}
The public API exposes two main workflows:
\begin{enumerate}
  \item \texttt{mach\_sweep}, which evaluates $T_w(M_{\infty})$ and
        dynamic pressure $q(M_{\infty})$ over a user-specified Mach
        range and altitude;
  \item \texttt{radome\_rf}, which computes transmission characteristics
        $T(f,T)$ for a given material model and wall thickness.
\end{enumerate}
Example scripts located in the \texttt{scripts/} directory demonstrate
how these functions can be combined to create complete parametric
studies and generate figures such as those shown in
Section~\ref{sec:results}. Basic unit tests in the \texttt{tests/}
directory verify consistency with analytic expressions for stagnation
temperature and transmission through a lossless quarter-wave transformer
in the isothermal limit.

The code is intentionally written in a clear, didactic style with
extensive inline documentation rather than micro-optimised for
performance. For the small problem sizes considered here (one or two
parametric dimensions), a typical sweep executes in well under a second
on a laptop, making the tool suitable for interactive exploration in
Jupyter notebooks or educational demonstrations.

\section{Case studies}
\label{sec:results}
We now illustrate typical uses of AeroRadomeSim through a set of simple
case studies. The goal is not to reproduce a specific flight vehicle,
but rather to highlight qualitative trends and the interplay between
aerothermal heating and RF transmission.

\subsection{Dynamic pressure and wall temperature versus Mach number}
As a first example we consider level flight at an altitude of
$h=10~\text{km}$, with free-stream properties obtained from a standard
atmosphere model and $\gamma = 1.4$. We evaluate dynamic pressure and
wall temperature using Eqs.~\eqref{eq:stagnationT} and
\eqref{eq:wallT}, with a representative recovery factor $r=0.9$.

Figure~\ref{fig:q_vs_mach} shows the resulting dynamic pressure as a
function of Mach number over the range $0 \le M_{\infty} \le 3$. The
approximately quadratic growth in $q$ with Mach number is evident. At
$M_{\infty} \approx 2.5$ the dynamic pressure exceeds
$5\times 10^5~\text{Pa}$, indicating substantial aerodynamic loading on
the radome structure and providing a convenient reference for sizing
studies.

\begin{figure}[t]
  \centering
  \includegraphics[width=0.7\textwidth]{docs/img/q_vs_mach.png}
  \caption{Dynamic pressure $q = \tfrac12\rho_{\infty} U_{\infty}^2$
  versus Mach number at $h=10~\text{km}$, generated with the
  \texttt{mach\_sweep} utility in \texttt{aeroradomesim}.}
  \label{fig:q_vs_mach}
\end{figure}

The same sweep also yields the wall temperature $T_w(M_{\infty})$,
which is used as input to the temperature-dependent dielectric model in
the next example. For brevity we omit an additional figure for
$T_w(M_{\infty})$, but it can be generated with a single additional
plot command in the provided example script.

\subsection{Radome transmission versus frequency at elevated temperature}
To illustrate the impact of temperature-dependent dielectric properties,
we consider a single-layer radome with thickness $d=10~\text{mm}$ and
baseline properties $\varepsilon'_0 = 4.0$ and
$\tan\delta_0 = 2\times 10^{-3}$ at $T_{\mathrm{ref}} = 300~\text{K}$.
We adopt modest positive thermal sensitivities
$a_{\varepsilon} = 2\times 10^{-3}~\text{K}^{-1}$ and
$a_{\delta} = 5\times 10^{-6}~\text{K}^{-1}$, representative of a
generic composite material.

Figure~\ref{fig:radome_s21} shows the magnitude of the transmission
coefficient $|S_{21}|$ (in decibels) as a function of frequency over the
L--S band at $T=300~\text{K}$. At this temperature the insertion loss is
modest, with $L_T$ below approximately $2~\text{dB}$ across the band for
normal incidence. As the Mach number and hence wall temperature
increase, Eqs.~\eqref{eq:epsT}--\eqref{eq:tandT} predict an increase in
both $\varepsilon'$ and $\tan\delta$, leading to stronger mismatch and
higher loss. The provided example script sweeps over temperature to
illustrate this trend; additional figures can be generated to visualise
$L_T$ as a function of both $f$ and $T$.

\begin{figure}[t]
  \centering
  \includegraphics[width=0.7\textwidth]{docs/img/radome_s21_T300K.png}
  \caption{Example radome transmission magnitude $|S_{21}|$ versus
  frequency at $T=300~\text{K}$ for a single-layer dielectric slab,
  computed with the transmission-line model in
  \texttt{aeroradomesim}.}
  \label{fig:radome_s21}
\end{figure}

These simple case studies demonstrate how AeroRadomeSim can be used to
explore the coupled effects of Mach number, altitude, material
properties, and wall thickness on both structural loading and RF
transmission using only a few lines of Python.

\section{Discussion, limitations, and future work}
The present model deliberately sacrifices geometric and physical
fidelity in favour of transparency and rapid turnaround. The main
limitations are:
\begin{itemize}
  \item The wall temperature is represented by a single equivalent value
        $T_w$, whereas real radomes exhibit spatially varying
        temperature distributions driven by three-dimensional flow and
        conduction through the structure.
  \item Only normal-incidence transmission through a single homogeneous
        layer is treated. Real systems may involve multi-layer
        sandwich constructions, frequency-selective surfaces, and
        oblique incidence.
  \item Temperature dependence of mechanical properties and structural
        deformation are not considered.
\end{itemize}
Despite these simplifications, the tool already supports several useful
applications in early-stage design, sensitivity analysis, and
education. Future work will focus on extending the RF module to handle
multi-layer stacks and oblique incidence, coupling AeroRadomeSim to
external aerothermal solvers that provide spatial temperature fields,
and adding convenience wrappers for Monte Carlo uncertainty studies
over material parameters.

\section*{Code availability}
The AeroRadomeSim source code is openly available on GitHub at:
\begin{center}
  \url{https://github.com/seyvars/AeroRadomeSim}
\end{center}
A frozen, citable snapshot of version~1.0.0 is archived on Zenodo under
the DOI:
\begin{center}
  \url{https://doi.org/10.5281/zenodo.17817671}
\end{center}

\bibliographystyle{plain}
\begin{thebibliography}{9}

% Replace the placeholders below with real references on radome design,
% aerothermodynamics, and transmission-line modelling before submission.

\bibitem{radome_text}
A.~Author.
\newblock \emph{Representative Radome Engineering Text}.
\newblock Publisher, Year.

\bibitem{aero_text}
B.~Author.
\newblock \emph{Compressible Flow and Aerothermodynamics}.
\newblock Publisher, Year.

\bibitem{em_text}
C.~Author.
\newblock \emph{Transmission-Line and Wave Propagation}.
\newblock Publisher, Year.

\end{thebibliography}

\end{document}
