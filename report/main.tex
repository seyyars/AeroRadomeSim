\documentclass[11pt,a4paper]{article}

\usepackage[margin=1in]{geometry}
\usepackage{amsmath,amssymb}
\usepackage{graphicx}
\usepackage{booktabs}
\usepackage[hidelinks]{hyperref}

\title{AeroRadomeSim: Numerical simulation of Mach 0--3 aerothermodynamics\\
and temperature-dependent radome RF transmission}

\author{Seyed Arsham Asgari%
\thanks{Independent researcher. Email: s.a.asg2000@gmail.com}}

\date{\today}

\begin{document}
\maketitle

\begin{abstract}
Airborne radomes experience significant aerothermodynamic heating at
moderate Mach numbers. The resulting temperature rise modifies the
temperature-dependent electromagnetic properties of the radome material
and degrades RF transmission. This work presents \emph{AeroRadomeSim},
a lightweight open-source Python tool that couples a simple
compressible-flow aerothermodynamic model with a temperature-dependent
RF transmission model for canonical radome configurations over Mach
numbers from $0$ to $3$. The software is intended for rapid parametric
studies, early-stage design screening, and educational use.
AeroRadomeSim is distributed under an open-source license, with source
code available on GitHub and a citable snapshot archived on Zenodo
(doi:\href{https://doi.org/10.5281/zenodo.17817671}{10.5281/zenodo.17817671}).
\end{abstract}

\section{Introduction}
Airborne antenna systems are commonly shielded from the external flow
by dielectric radomes. While the radome protects the antenna
mechanically, it also introduces RF transmission loss and boresight
error, both of which depend on the material permittivity, thickness,
and internal temperature. At flight conditions corresponding to Mach
numbers between $0$ and $3$, convective and compressible heating can
raise radome wall temperatures well above ambient conditions,
modifying the dielectric response of the material and, consequently, the
RF performance.

High-fidelity analysis of coupled aerothermodynamics and RF propagation
typically relies on large commercial CFD and full-wave electromagnetic
solvers. Such tools are powerful but also computationally expensive and
often inaccessible to students or early-stage design studies.
There is therefore value in small, script-level tools that implement
simplified but physically meaningful models and can be used for rapid
parametric exploration.

In this context, the main objectives of AeroRadomeSim are:
\begin{itemize}
  \item to provide a transparent implementation of a coupled
        aerothermodynamic and RF transmission model for canonical
        radome configurations over Mach 0--3;
  \item to enable fast parametric sweeps over Mach number, altitude,
        material properties, and wall thickness;
  \item to serve as a reproducible reference implementation that can be
        extended or embedded in larger analysis workflows.
\end{itemize}

\section{Physical and numerical model}

\subsection{Aerothermodynamic model}
The aerothermodynamic model in AeroRadomeSim is based on a
compressible-flow formulation with simple correlations for stagnation
or recovery temperature at the radome wall. Given a free-stream Mach
number $M_{\infty}$, static temperature $T_{\infty}$, and ratio of
specific heats $\gamma$, the adiabatic stagnation temperature is
\begin{equation}
  T_0 = T_{\infty} \left( 1 + \frac{\gamma - 1}{2} M_{\infty}^2 \right).
\end{equation}
A recoverable wall temperature can be expressed as
\begin{equation}
  T_w = T_{\infty} + r \left( T_0 - T_{\infty} \right),
\end{equation}
where $r$ is a recovery factor (typically between $0.8$ and $1.0$)
that may depend on the local boundary-layer regime.
For the purposes of this tool, we assume a spatially uniform equivalent
wall temperature $T_w$ representative of the most thermally loaded
region of the radome.

\subsection{Temperature-dependent RF transmission}
Given a radome wall of thickness $d$ and complex relative permittivity
$\varepsilon_r(T) = \varepsilon'(T) - \mathrm{j}\,\varepsilon''(T)$, the
normal-incidence power transmission coefficient $|T|^2$ of a
single-layer dielectric slab can be computed using standard
transmission-line expressions. AeroRadomeSim adopts a parametric model
for the temperature dependence of the real relative permittivity and
loss tangent:
\begin{align}
  \varepsilon'(T) &= \varepsilon'_0 + a_{\varepsilon} (T - T_{\mathrm{ref}}), \\
  \tan\delta(T)   &= \tan\delta_0 + a_{\delta} (T - T_{\mathrm{ref}}),
\end{align}
where $\varepsilon'_0$ and $\tan\delta_0$ are the properties at a
reference temperature $T_{\mathrm{ref}}$, and $a_{\varepsilon}$,
$a_{\delta}$ are user-specified linear coefficients.

\section{Software description}
AeroRadomeSim is implemented in Python and consists of small modules
that separate the aerothermodynamic model, RF transmission
calculations, and plotting utilities. The core dependencies are
\texttt{numpy}, \texttt{scipy}, and \texttt{matplotlib}. The source
tree includes example scripts and basic tests that demonstrate typical
usage patterns.

\section{Impact and reuse potential}
AeroRadomeSim is intentionally small, with a focus on clarity and
reproducibility. Anticipated use cases include early-stage design
studies, sensitivity and uncertainty analyses, and education. The code
is organized to facilitate extension: users can introduce alternative
correlations for wall temperature, more complex temperature
dependencies for permittivity, or multi-layer radome models while
reusing the existing workflow and plotting utilities.

\section*{Code availability}
The AeroRadomeSim source code is openly available on GitHub at:
\begin{center}
  \url{https://github.com/seyvars/AeroRadomeSim}
\end{center}
A frozen, citable snapshot of version~1.0.0 is archived on Zenodo under
the DOI:
\begin{center}
  \url{https://doi.org/10.5281/zenodo.17817671}
\end{center}

\end{document}
